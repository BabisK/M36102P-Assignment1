%%This is a very basic article template.
%%There is just one section and two subsections.
\documentclass[a4paper]{article}
\usepackage{a4wide}
\usepackage{amsmath}

\begin{document}

\title{Probability and Statistics for Data Analysis\\Assignment 1}
\author{Charalampos Kaidos}

\maketitle

\begin{enumerate}

  \item Assume that $A$ and $B$ are events of the sample space $S$ for which we
  know: $2P(A)-P(A')=3/5$, $P(B|A)=5/8$, $P(A|B)=4/9$. Calculate the following
  probabilities:
    \begin{enumerate}
      \item $P(A)$
        \begin{align}
          2P(A)-P(A')=3/5& \Leftrightarrow 2P(A)-(1-P(A))=3/5 \nonumber\\
          & \Leftrightarrow 2P(A)-1+P(A)=3/5 \nonumber\\
          & \Leftrightarrow 3P(A)=8/5 \nonumber\\
          & \Leftrightarrow P(A)=8/15
        \end{align}
      \item $P(A\cap B)$
        \begin{align}
          P(B|A)=\frac{P(A\cap B)}{P(A)}& \Leftrightarrow P(A\cap
          B)=P(B|A)P(A) \nonumber \\
          & \Leftrightarrow P(A\cap B)=5/8\times8/15=1/3
        \end{align}
      \item $P(B)$
        \begin{align}
          P(A|B)=\frac{P(A\cap B)}{P(B)}& \Leftrightarrow P(B)=\frac{P(A\cap
          B)}{P(A|B)}=\frac{1/3}{4/9}=9/12=3/4
        \end{align}
      \item $P(A\cup B)$
        \begin{align}
          P(A\cup B)=P(A)+P(B)-P(A\cap B)=8/15+3/4-1/3=171/180
        \end{align}
      \item Are the events $A$ and $B$ independent? \\
      Two events $A$ and $B$ are independent if $P(A\cap B)=P(A)P(B)$. For our
      problem:
      $$P(A)P(B)=8/15\times3/4=2/5\neq P(A\cap B)$$
      Thus the events $A$ and $B$ are not independent.
    \end{enumerate}
  \item Suppose that a sample space $S$ has $n$ elements. Prove that the number
  of subsets that can be formed from the elements of $S$ is $2^n$.
  \item Two players, A and B, alternatively and independently flip a coin and
  the first player to obtain a head wins. Assume player A flips first.
    \begin{enumerate}
      \item If the coin is fair, what is the probability that player A wins?
      \item More generally assume that $P(\text{head})=p$ (not necessarily
      $1/2$). What is the probability that player A wins?
      \item Show that $\forall p$ such that $0<p<1$, we have that $P(A\text{
      wins})>1/2$.
    \end{enumerate}
  \item A telegraph signals ``dot'' and ``dash'' sent in the proportion $3 : 4$,
  where erratic transmission cause a dot to become dash with probability $1/4$
  and a dash to become a dot with probability $1/3$.
    \begin{enumerate}
      \item If a dash is received, what is the probability that a dash has been
      sent?
      \item Assuming independence between signals, if the message dot-dot was
      received, what is the probability distribution of the four possible
      messages that could have been sent?
    \end{enumerate}
  \item Let $X$ be a continuous random variable with pdf $f(x)$ and cdf $F(x)$.
  For a fixed number $x_0$ (such that $F(x_0)<1$), define the function:
    \begin{align} g(x)=
      \begin{cases}
        \frac{f(x)}{1-F(x_0)} \quad &x\geq x_0 \\
        0 \quad &x<x_0
      \end{cases}
    \end{align}
  Prove that $g(x)$ is a pdf (also known as hazard function).
  \item A truncated discrete distribution is one in which a particular class
  cannot be observed and is eliminated from the sample space. In particular, if
  $X$ has range $0,1,2,\ldots$ and the $0$ class cannot be observed (as is
  usually the case), the $0$-truncated random variable $X_T$ has pmf:
  $$P(X_T=x)=\frac{P(X=x)}{P(X>0)}, \quad x=1,2,\ldots$$
  Find the pmf, mean and variance of the $0$-truncated $Poisson(\lambda)$ random
  variable.
\end{enumerate}

\end{document}
